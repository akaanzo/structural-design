
\chapter{Normative di riferimento}\label{chap:norme}
Le normative tecniche impiegate durante il progetto e la verifica degli elementi oggetto d'esame sono sia quelle nazionali, sia quelle europee corredate dal relativo annesso nazionale, con riferimento al progetto e alla verifica di elementi in cemento armato ordinario.

In particolare sono state utilizzate:
\begin{itemize}
	\item \texttt{Decreto 17 gennaio 2018 - Aggiornamento delle ``Norme tecniche per le\\ costruzioni''}, a cura del Ministero delle Infrastrutture e dei Trasporti;
	\item \texttt{Circolare 21 gennaio 2019, n. 7 C.S.LL.PP - Istruzioni per l'applicazione\\ dell'<<Aggiornamento delle ``Norme tecniche per le costruzioni''>> di cui al\\decreto ministeriale 17 gennaio 2018} - a cura del Ministero delle Infrastrutture e dei Trasporti;
	\item \texttt{EN 1992-1-1 (2004): Eurocode 2: Design of concrete structures - Part 1-1: General rules and rules for buildings} [Authoruty: The European Union Per Regulation 305/2011, Directive 98/34/EC, Directive 2004/18/EC].
	\item \texttt{Decreto 31 luglio 2012 - Approvazione delle Appendici nazionali recanti i\\parametri tecnici per l’applicazione degli Eurocodici.}\noindent\footnote{Per una miglior comporensione del testo e per una questione semplificativa d'ora in poi ci si riferir\`a al \texttt{Decreto 17 gennaio 2018} con l'acronimo \ntc, alla \texttt{Circolare 21 gennaio 2019, n. 7} con \circolare, l'eurocodice \texttt{EN 1992-1-1} con \ec e infine il \texttt{Decreto 31 luglio 2012} con \nad.}
\end{itemize}

Di seguito si riportano i capitolo specifici, a grandi linee, utilizzati durante i calcoli. In ogni caso, durante la stesura del seguente testo si farà riferimento a specifici paragrafi e tabelle di notevole rilevanza presenti nelle norme sopra elencate.

\section{Decreto 17 gennaio 2018 - \ntc}
Delle \ntc sono stati utilizzati i seguenti capitoli:
\begin{itemize}
	\item \textbf{2. Sicurezza e prestazioni attese} per il calcolo delle caratteristiche di sollecitazione del solaio (e precedentemente di trave e pilastri). In particolare le sezioni:
	\begin{itemize}
		\item \textbf{2.5. Azioni sulle costruzioni}, in particolare i paragrafi \textbf{2.5.2} e \textbf{2.5.3};
		\item \textbf{2.6. Azioni nelle verifiche agli stati limite} con riferimento alla \textbf{Tab. 2.6.I};
	\end{itemize}
	\item \textbf{3. Azioni sulle costruzioni} con  riferimento alla sezione \textbf{3.1. Opere civili e industriali} e quindi alle \textbf{Tab. 3.1.I}, \textbf{Tab. 3.1.II};
	\item \textbf{4. Costruzioni civili e industriali}, in particolare tutta la parte \textbf{4.1. Costruzioni in calcestruzzo};
	\textbf{11.2.10. Caratteristiche del calcestruzzo} per alcuni richiami sulle caratteristiche meccaniche del cls.
\end{itemize}

\section{Circolare 21 gennaio 2019, n. 7 - \circolare}
Riguardo alla \circolare sono state utilizzate le parti richiamete dalle \ntc nel paragrafo precendente, con particolare interesse al \textbf{capitolo C4. Costruzioni civili e industriali} parte \textbf{C4.1 Costruzioni in calcestruzzo}.

\section{EN 1992-1-1 - \ec + Decreto 31 luglio 2012 - \nad}
L'eurocodice 2 parte 1--1 verrà pesantemente utilizzato in tutte le sue parti, affiancato dal \textbf{NAD}, ogniqualvolta le normative nazionali non danno un sufficiente approfondimento del caso o rimandano alle normative europee.

\cleardoublepage

